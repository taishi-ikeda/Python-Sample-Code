\documentclass{jarticle}

\usepackage[letterpaper,top=2cm,bottom=2cm,left=3cm,right=3cm,marginparwidth=1.75cm]{geometry}
\usepackage{itembkbx}
\usepackage{boites,boites_exemples}
\usepackage{lipsum}
% Useful packages
\usepackage{amsmath}
\usepackage{graphicx}
\usepackage[colorlinks=true, allcolors=blue]{hyperref}
\usepackage{ascmac}
\usepackage[many]{tcolorbox}
\tcbuselibrary{breakable, skins, theorems}
\usepackage{color}
\usepackage{xcolor}
\usepackage{listings}
\lstset{%
  language={Python},
  basicstyle={\small},%
  identifierstyle={\small},%
  commentstyle={\small\itshape\color[rgb]{0,0.5,0}},%
  keywordstyle={\small\bfseries\color[rgb]{0,0,1}},%
  ndkeywordstyle={\small},%
  stringstyle={\small\ttfamily\color[rgb]{1,0,1}},
  frame={tb},
  breaklines=true,
  columns=[l]{fullflexible},%
  numbers=left,%
  xrightmargin=0zw,%
  xleftmargin=3zw,%
  numberstyle={\scriptsize},%
  stepnumber=1,
  numbersep=1zw,%
  lineskip=-0.5ex%
}
\title{モンキーハンティング問題}
\author{}
\date{}
\begin{document}
\maketitle
このコードは一様重力における1次元のモンキーハンティング問題を解くためのコードです。
モンキーハンティング問題とは、落下する物体Mを下から発射された弾丸Bを命中させる問題です。
今、$x-z$座標系を考え、-$z$方向に一様重力がかかっているとする。重力加速度を$g$とします。
弾丸Bの発射位置と初速度をそれぞれ$(x_{B0},z_{B0})$と$(v_{x0},,v_{z0}$とします。また物体Mの初期
位置を$(x_{M0},z_{M0})$として弾丸Bが発射されたタイミングで物体Mは静止しているとする。こ
の時、弾丸の軌道の式は$t$を時刻として
\begin{eqnarray}
x_{B}(t)&=&v_{x0}t+x_{B0}\\
z_{B}(t)&=&-\frac{g}{2}t^{2}+v_{z0}t+z_{B0}
\end{eqnarray}
で与えられます。一方で物体Mの軌道は
\begin{eqnarray}
x_{M}(t)&=&x_{M0}\\
z_{M}(t)&=&-\frac{g}{2}t^{2}+z_{M0}
\end{eqnarray}
故に弾丸Bと物体Mが衝突する条件は
\begin{eqnarray}
v_{x0}t+x_{B0}&=&x_{M0}\\
-\frac{g}{2}t^{2}+v_{z0}t+z_{B0}&=&-\frac{g}{2}t^{2}+z_{M0}
\end{eqnarray}
の二つの式が同時に満たされることで、得られる式は
\begin{eqnarray}
\frac{v_{z0}}{v_{x0}}=\frac{z_{M0}-z_{0}}{x_{M0}-x_{0}}
\end{eqnarray}
となります。
弾丸Bの発射時の角度を$\theta$とすると
\begin{eqnarray}
\tan\theta=\frac{v_{z0}}{v_{x0}}=\frac{z_{M0}-z_{0}}{x_{M0}-x_{0}}
\end{eqnarray}
を得ます。
それでは付属のコードのmain.pyを見てみます。
\begin{lstlisting}[caption=test,label=test]
import MonkeyHuntingClass as MH
g_acc=-9.80655 #重力加速度
#弾丸Bの初期位置
xb = 0.0 
zb = 0.0
#物体Mの初期位置
xm = 1.0
zm = 1.0
#弾丸の初速度の絶対値
v_abs=10.0

MH = MH.MonkeyHunting(g_acc)
MH.set_initial_bullet_position(xb,zb)
MH.set_initial_monkey_position(xm,zm)
MH.set_ground(0.0)
[flag,theta] = MH.solve_theta(v_abs)
if flag:
    print(theta)
else:
    print("Can not find the solution.")
 \end{lstlisting}
 一行目でMonkeyHunbtingクラスが定義されたMonkeyHuntingClass.pyというファイルを読み込んでいます。
 次に2行目から10行目までで使う変数を定義しています。
12行目でクラスを実体化しMHという名前にしています。
\end{document}
